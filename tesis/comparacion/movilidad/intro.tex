En la simulación de DTNs, los patrones de movilidad son fundamentales para emular el comportamiento de personas y paquetes de manera verosimil, lo que es importante para poder determinar cómo se comportará un protocolo en su contexto real de implementación.

La transmisión ocurre cuando los nodos se encuentran entre sí, intercambiando información de control y mensajes si los protocolos lo consideran apropiado. El tiempo que tarda un nodo en entrar en contacto con otro se denomina \emph{tiempo de encuentro}. El tiempo de contacto entre ellos, lo cual definirá las posibilidades de enviar datos entre sí mientras dure el contacto, se denomina \emph{tiempo de contacto}. Si no pueden concluir el intercambio de paquetes en ese tiempo, deben esperar hasta entrar nuevamente en contacto. Esto se denomina \emph{tiempo inter-encuentros}. Para realizar un análisis realista es necesario conocer estadísticamente estas tres cantidades. 

Los modelos de movilidad son generados utilizando trazas reales y modelos sintéticos con respaldo teórico, con lo que se busca lograr escenarios realistas. Generalmente, los distintos protocolos son evaluados utilizando ciertos modelos tradicionales, y se utilizan características especiales para afinar y demostrar mejoras en usos específicos de los protocolos, como redes vehiculares o de personas en contextos específicos. Se presentan algunos de los modelos más utilizados, clasificados en tres categorías.

La más simple y tradicional familia de modelos de movimiento son los de \emph{Random Mobility} o movimiento aleatorio, de los cuales destacan:

\begin{enumerate}
	\item Random Walk (RW)
	\item Random Walkway Point (RWP) 
	\item Random Direction (RD)
	\item Levy Walks (LW) 
\end{enumerate}

Otro tipo de modelos es derivado de patrones aleatorios, pero limitados a un mapa. En estos modelos los datos del mapa son incluídos utilizando el formato \emph{Well Known Text} (WKT), utilizando ampliamente en modelos de información geográfica (GIS). Algunos de los más populares modelos basados en mapas son:

\begin{enumerate}
	\item Map-Based Mobility Model (MBM)
	\item Shortest Path-Based Map (SPBM)
	\item Route-Based Map (RBM)
	\item Manhattan Mobility Model (MMM)
\end{enumerate}

En aquellas aplicaciones de DTNs donde los nodos son humanos, animales o vehículos que poseen ciertos patrones de movimiento, es posible explotar estos patrones para mejorar el desempeño de los protocolos. Estos patrones incluyen recorridos y modelos sociales basados en el comportamiento humano, por ejemplo:

\begin{enumerate}
	\item Community-Based Mobility Model (CMM)
	\item Time-Variant Community Model (TVCM)
	\item Working Day Movement (WDM)
	\item General Social Mobility Model (GeSoMo)
	\item Disaster Scenario (DS)
\end{enumerate}

Estos patrones de movilidad pueden ser configurados por cada nodo, por lo que es posible combinar varios modelos de movimiento en una simulación. A continuación se describe cada modelo en profundidad, analizando sus características y los escenarios que buscan representar.
