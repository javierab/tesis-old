\section{Modelos de Movilidad Aleatorios}

\subsection{Random Walk}
En el modelo Random Walk, cada nodo se mueve de manera impredecible, escogiendo un destino aleatorio y avanzando hacia él con dirección y velocidad de un rango predefinido. Una nueva dirección es asignada cada vez que el nodo llega a su destino. Cada nodo de la red se comporta de manera independiente a los otros. Este modelo no mantiene memoria de la movilidad anterior, ya que no lleva registro de los patrones formados por la ubicación y movimiento de los nodos, siendo su movimiento independiente de su historia. Esto puede generar algunos movimientos poco realistas como paradas abruptas o vueltas cerradas. A pesar de ello, este modelo ha sido popular al evaluar protocolos DTN por su estabilidad en la distribución de los parámetros de movilidad por comportarse de manera estable.

\subsection{Random Walkway Point}
Este es otro modelo básico y simple utilizado frecuentemente en simulaciones. RWP incluye una pausa después de concluir un segmento de caminata aleatorio. La dirección, velocidad y tiempos de pausa son tomados de una distribución uniforme. Un nodo se mueve a un destino a velocidad constante, se queda ahí por un momento, y luego toma un nuevo destino aleatoriamente. Este modelo es fácil de analizar y de implementar, por lo que es también ampliamente utilizado para evaluar protocolos a pesar de su movimiento artificial. Se presenta a veces un problema de distribución no uniforme en los nodos, lo cual complica su análisis y su uso.

\subsection{Random Direction}
Este modelo de distribución aleatoria de direción soluciona el problema encontrado en RWP. En este modelo, un nodo obtiene un ángulo aleatorio y se mueve en esa dirección hasta encontrar la frontera del área de simulación. Al encontrarla, el nodo se detiene por un tiempo específico y luego escoge una nueva dirección, moviéndose nuevamente. Este modelo es más consistente que otros modelos aleatorios, y ofrece distribuye uniformemente los nodos en el área de simulación. Existen muchas variantes de RD, donde se incorporan factores para manejar la duración de la caminata y diferentes efecto al llegar a la frontera.

\subsection{Levy Walks}
Este modelo es similar a RW, pero donde las variables de movilidad son tomadas de una distribución de ley potencial. Este modelo es capaz de generar distribuciones de tiempo inter-contacto similar a trazas reales, pero sin capturar las relaciones entre nodos. Otra característica es la alta \emph{difusividad}, es decir, el desplazamiento en el tiempo tiene una alta varianza (16). Una variante de este modelo es Truncated Levy Walks, que utiliza distribuciones truncadas de Paretto para emular la movilidad en una área confinada y distribuciones truncadas de Paretto para áreas externas. TLW entrega representaciones más realistas de los patrones estadísticos encontrados en la movilidad humana mientras preserva la simplicidad de los modelos aleatorios.
