%orientado a sensores-sink, tipo spray-and-focus con R copias, utility function
\subsubsection{SCAR}
El protocolo SCAR\cite{scar}, orientado a sensores de baja independencia energética, busca tomar información del contexto del sensor (como vecinos encontrados o nivel de batería) para prever cuáles nodos son los mejores para llevar los datos. A pesar de realizar varias réplicas, considerando la posible desconexión de uno o más nodos, el número es considerablemente menor que en las estrategias epidémicas. Se toma como hipótesis que el destino es siempre un nodo de un conjunto de sumideros.

Este protocolo calcula constantemente sus cambios en conectividad, posición respecto a sumideros y nivel de batería, los cuales se combinan para calcular localmente la probabilidad de entrega a los sumideros, $P(s_i)$ en cada sensor $s_i$.

Un nodo $A$ sólo transfiere datos a otro $B$ si la probabilidad de entrega de $B$ es mayor que la de $A$, entregando $R$ copias de sus datos a los $R$ nodos con mayor probabilidad de entrega ---donde puede estar él mismo--- marcando la copia de mayor probabilidad como $master$ y el resto como $backup$. Las copias $backup$, al ser sólo una alternativa, pueden ser eliminadas si un nodo requiere liberar espacio de almacenamiento, mientras que $master$ sólo puede borrarse si, al llegar a un sumidero, se le notifica al nodo que el mensaje ya ha sido previamente entregado. Estas $R$ copias, a su vez, pueden ser \emph{forwardeadas} a otros nodos si tienen una mejor probabilidad de entrega, pero no se crean nuevas copias.

Esta estrategia podría generar acumuladores de datos si ciertos nodos tienen una probabilidad muy alta de entrega, pero esto genera un consumo mayor de energía y por ende un menor nivel de batería, lo cual es tomado en consideración en la probabilidad de entrega buscando precisamente evitar ese escenario.

Este protocolo, a pesar de su complejidad, sólo fue comparado en \cite{scar2} con una entrega epidémica \emph{random}, sin resultados notables, concluyendo que le ocurre algo similar a Spray-and-Focus, donde la complejidad del cálculo de la probabilidad de entrega juega en contra de la simpleza de la replicación.