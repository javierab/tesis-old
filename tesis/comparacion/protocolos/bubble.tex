%forwarding, social, hierarchical
\subsubsection{BUBBLE}
BUBBLE \cite{bubble} es un protocolo social, donde en vez de considerar los patrones de movilidad de los nodos busca evaluar sus relaciones e importancia en la red, la cual es menos cambiante que su patrón de movimiento. Para realizar esto, BUBBLE toma dos parámetros: las \emph{comunidades} formadas por la cooperación entre nodos, y la \emph{centralidad} de un nodo dentro de su comunidad, es decir, qué tan popular es, determinado por el número de interacciones de éste, que también logra capturar el rol que puede tener un nodo en la red.

El protocolo requiere tres valores de cada nodo: su comunidad $c$, su ranking local $l$ en su comunidad, y su ranking global $g$ entre todos los nodos del sistema. Al enviar un mensaje $m$, el protocolo evalúa si ya encontró la comunidad de destino. Si es así, intenta avanzar por los nodos de mayor $l$. En caso contrario, intenta avanzar por los nodos de mayor $g$, hasta que el mensaje llegue al destino o alcance su timeout. 

Una dificultad de este algoritmo está en evaluar la existencia de comunidades y centralidad en éstas. Para ello se utilizan dos algoritmos de detección de comunidades: \texttt{K-CLIQUE} de Palla et al. y \emph{Weighted Network Analysis} (WNA) de Newman \cite{bubble24}. \texttt{K-CLIQUE} puede reconocer comunidades superpuestas, pero al ser para grafos binarios no puede asignar pesos a los enlaces. Por su parte, WNA puede asignar pesos pero no encontrar superposiciones, por lo que se combinan ambos. Este análisis, además, debe hacerse de manera distribuida, donde cada nodo debe encontrar su comunidad y calcular su centralidad, para lo que se introduce DiBuBB, algoritmo distribuido que aproxima estos valores, encontrando correlación en el número de nodos únicos visto en una unidad de tiempo con la centralidad, siendo por tanto un buen estimado.

BUBBLE logra buenos resultados en simulaciones utilizando varias trazas de movimientos humanos reales, obteniendo resultados similares a PRoPHET en tasa de entrega, pero con menor \emph{overhead}. BUBBLE promete obtener buenos resultados tanto en comunidades de estructura plana, como las de prueba, como en estructura jerárquica, que se adapta directamente a la centralidad social.
