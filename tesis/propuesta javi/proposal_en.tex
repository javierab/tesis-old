\documentclass[10pt,preprint,onecolumn]{paper}
\usepackage[utf8]{inputenc}
\usepackage[english]{babel}
\usepackage[T1]{fontenc}
\usepackage{mathtools, comment, url, hyperref}
\usepackage[cm]{fullpage}
\usepackage{setspace}
\usepackage{float}
\usepackage{wrapfig}
\usepackage[pdftex]{graphicx}
\usepackage{tikz}
\usepackage{caption}
\usepackage{subcaption}
\restylefloat{figure}
\usepackage[pdftex]{graphicx}
\usepackage{tikz}
\usetikzlibrary{trees}
\usepackage{caption}
\usepackage{cite}
\usepackage{subcaption}
\usepackage{lscape}
\singlespacing

\title{Routing in partially connected wireless sensor networks: Modelling and implementation of an algorithm for prioritization in mining topologies.}
\author{Javiera Born}

\begin{document}
\begin{abstract}

\end{abstract}

Wireless networks, since its massification in the 90s, have revolutionized the way communication is conceived. Specifically, wireless sensor networks have become one of the most [con futuro] because of the huge variety of applications they can offer, covering topics such as industrial safety, personal health and fast emergency response. These concepts can be particularly applied to mining, where sensors may be used to increase safety of people in tough environments via rapidly communicating possible risks in the short term, and mining data obtained from sensors to predict long-term issues.

Some difficulties arise when routing in underground mines, for there is little or inexisting infrastructure since the costs of implementing and maintaining it are excessively high, and some zones may be abandoned after its exploitation. This brings up the idea of using the operators themselves to route data creating a so called \emph{ad-hoc} network between them, and also including connected machinery.

In this proposal we look forward mixing up the constant recollection of sensed data with a routing prioritization of certain messages, destined to automatic alerts and vocal communication in the context of an intermittently connected topology, where establishing a routing route may be difficult since connectivity between nodes changes, and paths change with them, causing typical routing algorithms to fail.

To archive an efficient routing, a deep exploration of the routing protocols and mobility models used for disrupt-tolerant networks is done. For properly applying them in the mining environment, some hypothesis are made about the communication needs and possibilities. A new protocol will then be proposed to fit into the mining context, which will be tested using OMNeT++ simulator, considering various topologies and mobility patterns to properly test the protocol performance in realistic scenarios.

Finally, we consider a real implementation of the routing protocol using sensor nodes in a mining context to validate the simulation results and assumptions taken in the process of design and evaluation.

\newpage

\section{Work Done}
\begin{landscape}

\begin{figure}[H]
\begin{tikzpicture}
  [auto,every node/.style={rectangle,draw, text centered, text width=2.0cm,minimum height=1.5cm },node distance=4cm]
\tikzset{%
level 1/.style={sibling distance = 11.1cm, level distance=2cm,edge from parent path={(\tikzparentnode.south) -- (\tikzchildnode.north)}},
level 2/.style={sibling distance = 7.5cm, level distance=2cm,edge from parent path={(\tikzparentnode.south) -- (\tikzchildnode.north)}},
level 3/.style={sibling distance = 2.5cm, level distance=2cm,edge from parent path={(\tikzparentnode.south) -- (\tikzchildnode.north)}},
level 4/.style={sibling distance = 2.5cm, level distance=2cm,edge from parent path={(\tikzparentnode.south) -- (\tikzchildnode.north)}},
level 5/.style={sibling distance = 2.5cm, level distance=2cm,edge from parent path={(\tikzparentnode.south) -- (\tikzchildnode.north)}},
level 6/.style={sibling distance = 2.5cm, level distance=2cm,edge from parent path={(\tikzparentnode.south) -- (\tikzchildnode.north)}},
}

\node (0){DTN}
    child {node (1) {Replicative}
              child {node (2) {Controlled \\ Copies #}
                child{node (3) {No additional \\ info}
                    child[blue]{node (4) {Spray and \\Wait}}}
                child{node {Utility\\ function}
                    child[blue]{node {Spray and Focus}
                    child{node {SCAR}
                    child{node {CRHC}}}}}
                child{node {Encounters\\ history}
                    child[blue]{node{PRoPHET}}}
                child{node {Probabilistic \\ estimation}
                    child[blue]{node{Plankton}}}}
              child {node {Unlimited \\Copies}
                child{node {No additional \\ info}
                    child[blue]{node {Epidemic}}}
                child{node {Probabilistic \\ estimation}
                    child[blue]{node {MaxProp}
                    child{node {RAPID}}}}}}
    child {node {Forwarding}
              child {node (5){Encounters \\ history}
                child[blue] {node (6){PER}
                child {node {SMART}}}}
              child {node (5){Social and \\mobility info}
                child[blue] {node (6){SOLAR}
                child {node {BUBBLE}}}}};
\end{tikzpicture}
\label{fig:tax}
\end{figure}
\end{landscape}
\newpage
A protocol taxonomy and comparison has been made (see Figure \ref{fig:tax})

\end{document}